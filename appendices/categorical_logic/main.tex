\refstepcounter{chapter}%
\chapter*{\thechapter \quad Categorical Logic}
As mentioned in the previous chapter,
knowledge base embeddings are maps that approximately preserves the structure of the knowledge base.
We hinted towards the idea that 'preserving the structure' entails
that
\begin{gather*}
    \mathcal{K}\vdash C\sqsubseteq D \implies \mathcal{D}(E(C),E(D))\approx 1\\
    \mathcal{K}\vdash C\not\sqsubseteq D \implies \mathcal{D}(E(C),E(D))\approx 0
\end{gather*}
where $\mathcal{D}$ is the decoding function.
This assumption would mean that $\mathcal{D}$ preserves the subsumption relation of the knowledge base
'in some way'.

Notice that 'in some way' is not a precise criteria. While we can compute 'how much' 
$\mathcal{D}$ preserves the subsumption relation of the knowledge base,
we would like to understand what it means for a map to preserve the subsumption relation of the knowledge base.
This is crucial both in order to be able to calculate what an optimal $\mathcal{D}$
should be and to be able to understand what are the constraints that we should impose on $\mathcal{D}$.

Not only that, but we might wish to explore other properties of the knowledge base that we would like to preserve.
For instance, we might want to explore the possibility that $E$
preserves the constructors of the description logic. E.g.
\begin{gather*}
    \mathcal{K}\vdash B\sqsubseteq C\sqcap D \implies \mathcal{D}(E(B),E(C))\approx 1\text{ and } \mathcal{D}(E(B),E(D))\approx 1
    \\
    \mathcal{K}\vdash B\sqcup C\sqsubseteq D \implies \mathcal{D}(E(B),E(C))\approx 1\text{ and } \mathcal{D}(E(B),E(D))\approx 1
    \\
    \vdots
\end{gather*}
Understanding what it means for a map to preserve the structure of a knowledge base
would allow us to prceisely define the constraints that we should impose on 
an embedding and the decoding function in order to preserve the structure of the knowledge base
that we are trying to embed.

Category theory provides a framework to discuss questions of structure preservence of maps.
We would develop the core ideas relevant to this thesis in this chapter.

\section{Categories}
As mentioned above, category theory provides a framework to discuss questions of 
structure preservence of maps.
In this section we shall introduce the core ideas of category theory.
In foundational mathematics,two objects 
are considered to be the equal if they are made of the same 
elements.

For instance, two groups $(G,\circ_1), (H,\circ_2)$ are considered to be equal if
$G=H$ and $\circ_1=\circ_2$.
Most mathematicians would consider this condition
to be too strong. Most mathematicians would consider two 
groups to be 'equal' (\it{isomorphic}) if they behave the same way.
I.e. if there is a bijection $f:G\to H$ such that
\[
    f(g_1\circ_1 g_2)=f(g_1)\circ_2 f(g_2)
\]
for all $g_1,g_2\in G$.
This means that the elements of $G$ and $H$ are not 
necessarily the same,
but they behave the same way.
'Behaving the same way' can mean 
different things in different domains of mathematics.
Each domain of mathematics
can be seen as a category $\mathcal{C}$ which consists
of the collection mathematical objects $\mathcal{C}_0$ of interest and 
the collection of maps $\mathcal{C}_1$ 
between them are the maps that preserve
the structure of the objects.
Given two objects $A,B\in\mathcal{C}_0$,
we denote the set of maps $f:A\to B$
by $\mathcal{C}(A,B)$.

As the maps preserve the structure of the objects, meaning
that they do not break the structure of the objects,
we should be able to compose them.
Hence a category should come equipped with a composition
operator
\[
    \circ:\mathcal{C}_1\times\mathcal{C}_1\to\mathcal{C}_1
\]
such that given $f:A\to B$ and $g:B\to C$,
we have $g\circ f:A\to C$.

As categories express the meaning of 'behaving the same way',
we should be able to distinguish
between objects that are identical and objects that are 
isomorphic.
For every object $A\in\mathcal{C}_0$,
there should be an identity map $\mathbf{1}_A:A\to A$.
The identity map should be the identity element of 
the composition operator as well i.e.
\[
    \mathbf{1}_A\circ f=f\circ \mathbf{1}_A=f
\]
for all $f:A\to B$.

We now have all the ingredients to define a category.

\begin{definition}[Category]
    Given a collection of objects $\mathcal{C}_0$ and a collection of maps $\mathcal{C}_1\subseteq \mathcal{C}_0\times\mathcal{C}_0$,
    a category $\mathcal{C}$ is a tuple $(\mathcal{C}_0,\mathcal{C}_1,\circ)$ where
    \begin{itemize}
        \item[compositionality] $\circ:\mathcal{C}_1\times\mathcal{C}_1\to\mathcal{C}_1$ is a
        map such that for every $f:A\to B$ and $g:B\to C$, we have $g\circ f:A\to C$.
        \item[existence of identity] For every $A\in\mathcal{C}_0$, there exists an identity map $\mathbf{1}_A:A\to A$
        such that for every $f:A\to B$, we have $f\circ \mathbf{1}_A=f$ and $\mathbf{1}_B\circ f=f$.
        \item[associativity] For every $f:A\to B$ and $g:B\to C$ and $h:C\to D$, we have $h\circ (g\circ f)=(h\circ g)\circ f$.
    \end{itemize}
\end{definition}

Two objects $A,B\in\mathcal{C}_0$ are isomorphic if
for any property, relevant to the specific domain of mathematics,
that $A$ satisfies, $B$ satisfies as well.
In category theory, this translates
to the existence of two maps $f:A\to B$ and $f^{-1}:B\to A$
such that $f^{-1}\circ f=\mathbf{1}_A$ and $f\circ f^{-1}=\mathbf{1}_B$.
We call $f$ the isomorphism between $A$ and $B$ and
we call $f^{-1}$ the inverse of $f$.
We denote the set of isomorphisms between $A$ and $B$ by
$A\cong B$.

As this thesis focuses on the structure of knowledge bases,
we would be interested in categories that are relevant for logic. We provide examples of such categories.

\section{Lattices}
A very simple example of a category is a post $(P,\leq)$. 
The elemenets of a poset are the objects of the category and the arrows (maps) are the order relations.
The transitivity of the order relation implies that the composition of two arrows is an arrow.
The reflexivity of $\leq$ implies that the identity map is the arrow $a\leq a$.