\refstepcounter{chapter}%
\chapter*{\thechapter \quad Categorical Logic}
As mentioned in the previous chapter,
knowledge base embeddings are maps that approximately preserves the structure of the knowledge base.
We hinted towards the idea that 'preserving the structure' entails
that
\begin{gather*}
    \mathcal{K}\vdash C\sqsubseteq D \implies \mathcal{D}(E(C),E(D))\approx 1\\
    \mathcal{K}\vdash C\not\sqsubseteq D \implies \mathcal{D}(E(C),E(D))\approx 0
\end{gather*}
where $\mathcal{D}$ is the decoding function.
This assumption would mean that $\mathcal{D}$ preserves the subsumption relation of the knowledge base
'in some way'.

Notice that 'in some way' is not a precise criteria. While we can compute 'how much' 
$\mathcal{D}$ preserves the subsumption relation of the knowledge base using the expected loss,
we would like to understand what it means for a map to preserve the subsumption relation of the knowledge base.
This is crucial both in order to find an optimal $\mathcal{D}$
and to be able to understand what are the constraints that we should impose on $\mathcal{D}$.

Not only that, but we might wish to explore other properties of the knowledge base that we would like to preserve.
For instance, we might want to explore the possibility that $E$
preserves the constructors of the description logic. E.g.
\begin{gather*}
    \mathcal{K}\vdash B\sqsubseteq C\sqcap D \implies \mathcal{D}(E(B),E(C))\approx 1\text{ and } \mathcal{D}(E(B),E(D))\approx 1
    \\
    \mathcal{K}\vdash B\sqcup C\sqsubseteq D \implies \mathcal{D}(E(B),E(C))\approx 1\text{ and } \mathcal{D}(E(B),E(D))\approx 1
    \\
    \vdots
\end{gather*}
Understanding what it means for a map to preserve the structure of a knowledge base
would allow us to prceisely define the constraints that we should impose on 
an embedding and the decoding function in order to preserve the structure of the knowledge base
that we are trying to embed.

Category theory provides a framework to discuss questions of structure preservence of maps.
We would develop the core ideas relevant to this thesis in this chapter.

\section{Categories and Algebraic Logic}\label{sec:cat_alg_logic}
In this section we shall develop the core ideas of category theory and basic objects in the field
by working through the example of algebraic logic.
Algebraic logic is a branch of mathematical logic that represents logics as algebraic structures.
This allows us to use rigorous mathematical tools to study the properties of logics.
The properties of a logic can often be understood by 
the provability relation $\vdash$ of the logic.

In mathematics, it is often important to distinguish between equality and isomorphism.
Equality is often interpreted as Leibniz's principle of indiscernibility of identicals.
That is, two objects $A$ and $B$ are equal if for any property $P$, $P(A)\iff P(B)$.
While this property is useful for explicitly constructing objects,
it is often too restrictive for discussing the properties of objects under
a given mathematical theory. For instance, set theory
provides us with multiple ways of constructing a model of the natural numbers.
While these constructions are useful for showing that we can construct the natural numbers,
they are not equal to one another, they have different properties.
When we wish to explore number theoratic or algebraic properties of the natural numbers,
we woud restrict to properties that are relevant for such topics. Thus, our
models of the natural numbers would behave the same under those properties. This notion
of sameness is called isomorphism. The collection of objects that we are concerned
with is called a category. For instance the cateogry of algebraic structures.

It is usually not practical to classify all the properties that we wish our objects to be invariant under,
hence we need further assumptions to explore what it means for two objects to be structurally the same.
Category theory is considered a structuralist approach to mathematics.
In this approach, mathematical objects are defined by their relations to other objects
in the same category. 
Relations here are understood as maps between objects that respect the structure of the objects.
For instance, in a logic, the falsum object $\bot$ is defined by the fact that for any
other proposition $p$, $\bot\vdash p$.

Under this approach, two objects are isomorphic if they relate to all objects in the category
in the same way. For instance, given two propositions $p$ and $q$, we say that $p$ and $q$
are isomorphic (denoted by $p\cong q$) if for any proposition $r$, $p\vdash r$ if and only if $q\vdash r$.
Likewise, in the category of sets $\mathbf{Set}$, two sets $A$ and $B$ are isomorphic if and only if
for any set $C$, there exists a bijection between the set of maps from $A$ to $C$ and the set of maps from $B$ to $C$.
This essentially boils down to the fact that $A$ and $B$ are in bijection.

Hence, a category $\mathcal{C}$ consists of a collection of objects $\mathcal{C}_0$
and a collection of maps $\mathcal{C}_1$ between them. Given two objects $A,B\in\mathcal{C}_0$,
we denote the collection of maps between them by $\mathcal{C}(A,B)$ and call it the hom-set of $A$ and $B$.
Since isomorphism in the category of sets is bijection, 
given any category $\mathcal{C}$, and any two objects $A,B\in\mathcal{C}_0$,
$A\cong B$ is the same as the claim that for any object $C\in\mathcal{C}_0$,
$\mathcal{C}(A,C)\cong \mathcal{C}(B,C)$.

Isomorphism itself can be understood as a structure preserving map within the category.
Primarily, every object $p$ in a category is isomorphic to itself through a unique
map called the identity map $\mathbf{1}_p:p\to p$.
For any two objects $p,q$ in a category, their isomorphism
is a map $f:p\to q$ such that there exists an inverse map $f^{-1}:q\to p$ such that
$f^{-1}\circ f=\mathbf{1}_p$ and $f\circ f^{-1}=\mathbf{1}_q$.

As the maps preserve the structure of the objects, meaning
that they do not break the structure of the objects,
we should be able to compose them.
Hence a category should come equipped with a composition
operator
\[
    \circ:\mathcal{C}_1\times\mathcal{C}_1\to\mathcal{C}_1
\]
such that given $f:A\to B$ and $g:B\to C$,
we have $g\circ f:A\to C$.

We now have all the ingredients to define a category.

\begin{definition}[Category]
    Given a collection of objects $\mathcal{C}_0$ and a collection of maps $\mathcal{C}_1\subseteq \mathcal{C}_0\times\mathcal{C}_0$,
    a category $\mathcal{C}$ is a tuple $(\mathcal{C}_0,\mathcal{C}_1,\circ)$ where
    \begin{itemize}
        \item[compositionality] $\circ:\mathcal{C}_1\times\mathcal{C}_1\to\mathcal{C}_1$ is a
        map such that for every $f:A\to B$ and $g:B\to C$, we have $g\circ f:A\to C$.
        \item[existence of identity] For every $A\in\mathcal{C}_0$, there exists an identity map $\mathbf{1}_A:A\to A$
        such that for every $f:A\to B$, we have $f\circ \mathbf{1}_A=f$ and $\mathbf{1}_B\circ f=f$.
        \item[associativity] For every $f:A\to B$ and $g:B\to C$ and $h:C\to D$, we have $h\circ (g\circ f)=(h\circ g)\circ f$.
    \end{itemize}
\end{definition}

\begin{example}[Category of sets]
    The category of sets $\mathbf{Set}$ is a category such that
    \begin{itemize}
        \item[objects] The objects of $\mathbf{Set}$ are the sets
        \item[morphisms] The morphisms of $\mathbf{Set}$ are the functions between sets
        \item[composition] The composition of two maps $f:A\to B$ and $g:B\to C$ is the reguar composition of functions $g\circ f:A\to C$
        \item[identity] The identity map of a set $A$ is the map $\mathbf{1}_A:A\to A$ such that for any $a\in A$, $\mathbf{1}_A(a)=a$
    \end{itemize}    
\end{example}

\begin{definition}[Subcategory]
    Given two categories $\mathcal{C}$ and $\mathcal{D}$, a subcategory $\mathcal{D}$ of $\mathcal{C}$ is a pair of subsets
    $\mathcal{D}_0\subseteq \mathcal{C}_0$ and 
    $\mathcal{D}_1\subseteq \mathcal{C}_1$
    such that 
    \begin{itemize}
        \item For every $A\in\mathcal{D}_0$, $\mathbf{1}_A\in\mathcal{D}_1$
        \item For every $f:A\to B\in\mathcal{D}_1$, $A,B\in\mathcal{D}_0$
        \item For every $f:A\to B\in\mathcal{D}_1$ and $g:B\to C\in\mathcal{D}_1$, $g\circ f\in\mathcal{D}_1$
    \end{itemize}
\end{definition}

A very simple example of a category is a poset $(P,\leq)$. 
The elemenets of a poset are the objects of the category and the arrows (maps) are the order relations.
The transitivity of the order relation implies that the composition of two arrows is an arrow.
The reflexivity of $\leq$ implies that the identity map is the arrow $a\leq a$.

Notice that the order relation of a poset behaves 
like the provability relation $\vdash$ of propositional logic.
That is, if $p\vdash q$ and $q\vdash t$, then $p\vdash t$,
$p$ is always a proof of $p$ itself and so on.

However, logic systems usually have more structure than just provability relation.
For instance, most logic systems have a top and a bottom element, 
denoted by $\top$ and $\bot$ respectively.
These two elemenets could be defined through the implication relation as
the two elements for which, for all $p$, $p\vdash \top$ and $\bot\vdash p$.
In a category, such elements are called
\begin{definition}[Initial and Terminal Objects]
    Given a category $\mathcal{C}$, an object $A\in\mathcal{C}_0$ is called initial if for every $B\in\mathcal{C}_0$,
    there exists a unique map $f:A\to B$.
    An object $A\in\mathcal{C}_0$ is called terminal if for every $B\in\mathcal{C}_0$,
    there exists a unique map $f:B\to A$.
    We denote these objects by $\bot$ and $\top$ respectively.
\end{definition}
A poset that has a top and a bottom element is called a bounded poset.

Likewise, we can define the conjunction and disjunction of two elements
through the provability relation as follows:
\begin{definition}[Binary meets and join]
    For any two elements $p,q$ of a poset $(P,\leq)$, we define
    the meet $p\wedge q$ and the join $p\vee q$ as the elements such that for any $t\in P$,
    \begin{gather*}
        t\leq p\wedge q \iff t\leq p\text{ and }t\leq q\\
        p\vee q\leq t \iff p\leq t\text{ and }q\leq t
    \end{gather*}
\end{definition}
A poset in which every two elements have a meet (join) is called a meet (join) semilattice.
A poset that is both a meet and a join semilattice is called a lattice.
In cateogircal terms, a meet (join) semilattice is a category in which every two objects have a product (coproduct).

\begin{definition}[Products and coproducts]
    Given a category $\mathcal{C}$ and two objects $A,B\in\mathcal{C}_0$,
    a product of $A$ and $B$ is an object $A\times B$ and two maps
    $\pi_1:A\times B\to A$ and $\pi_2:A\times B\to B$ such that for any object $C\in\mathcal{C}_0$
    and two maps $f:C\to A$ and $g:C\to B$, there exists a unique map $h:C\to A\times B$ such that
    \[
        \pi_1\circ h=f\text{ and }\pi_2\circ h=g
    \]
    A coproduct of $A$ and $B$ is an object $A\sqcup B$ and two maps
    $\iota_1:A\to A\sqcup B$ and $\iota_2:B\to A\sqcup B$ such that for any object $C\in\mathcal{C}_0$
    and two maps $f:A\to C$ and $g:B\to C$, there exists a unique map $h:A\sqcup B\to C$ such that
    \[
        h\circ \iota_1=f\text{ and }h\circ \iota_2=g
    \]
    Alternatively,
    the product of two objects $A$ and $B$ is an object such that
$\mathcal{C}(C,A\times B)\cong \mathcal{C}(C,A)\times \mathcal{C}(C,B)$
and the coproduct of two objects $A$ and $B$ is an object such that
$\mathcal{C}(A\sqcup B,C)\cong \mathcal{C}(A,C)\times \mathcal{C}(B,C)$.
Described in such a way, we can see how similar are the meets and joins to the products and coproducts.

\end{definition}

The existence of binary meets (joins) and the top (bottom) element implies that every finite
subset of the semilattice has a meet (join).
We can define the meet (join) of an arbitrary subset of the semilattice
as the element such that for any $t\in P$,
\begin{gather*}
    t\leq \bigwedge S \iff \forall p\in S, t\leq p\\
    \bigvee S\leq t \iff \forall p\in S, p\leq t
\end{gather*}
If every subset of the semilattice has a meet (join), then the semilattice is called a complete meet (join) semilattice.

An arbitrary product (coproduct) of a set of objects is defined similarly as an indexed set of maps
$\{\pi_i:A_i\to A\}_{i\in I}$ ($\{\iota_i:A\to A_i\}_{i\in I}$) such that for any object $B\in\mathcal{C}_0$
and a set of maps $\{f_i:B\to A_i\}_{i\in I}$ ($\{g_i:A_i\to B\}_{i\in I}$), there exists a unique map $h:B\to A$ ($h:A\to B$) such that
\[
    \pi_i\circ h=f_i\text{ and }h\circ \iota_i=g_i
\]
for all $i\in I$.
A category in which every two objects have a product (coproduct) is called a cartesian (cocartesian) category.

Lastly, we can define the implication relation through the meet and the join as follows:
\begin{definition}[Implication]
    Given a poset $(P,\leq)$ and two elements $p,q\in P$, we define the implication $p\Rightarrow q$ as
    an object such that for any $t\in P$,
    \[
        t\leq p\Rightarrow q \iff t\wedge p\leq q
    \]
\end{definition}
If $\bigvee \{t:t\wedge p\leq q\}$ exists, then it is equal to $p\Rightarrow q$.
In categorical terms, the implication relation is the exponential of two objects.
\begin{definition}[Exponential]
    Given a category $\mathcal{C}$ and two objects $A,B\in\mathcal{C}_0$,
    an exponential of $A$ and $B$ is an object $B^A$ and a map $\epsilon:B^A\times A\to B$ such that for any object $C\in\mathcal{C}_0$
    and a map $f:C\times A\to B$, there exists a unique map $g:C\to B^A$ such that
    \[
        \epsilon\circ (g\times \mathbf{1}_A)=f
    \]
    Alternatively we can define it through the hom-set as the unique object $B^A$ such that
    for any object $C\in\mathcal{C}_0$, we have
    \[
        \mathcal{C}(C\times A,B)\cong \mathcal{C}(C,B^A)
    \]
\end{definition}
A category with a terminal object such that
all pairs of objects have an exponential object and a product is called a cartesian closed category.

Lastly, we can define the negation of an element through the implication relation:
$\neg p:=p\Rightarrow \bot$.
A bounded lattice in which every two elements have an implication $a\Rightarrow b$ is called a Heyting algebra.
A Heyting algebra that also satisfies the law of excluded middle $a\vee \neg a=\top$ is called a Boolean algebra.
These two types of algebras are the algebraic counterparts of the intuitionistic and classical propositional logic respectively.
\begin{definition}[Heyting and Boolean algebras]
    A Heyting algebra is a structure $(H,\leq,\wedge,\vee,\Rightarrow,\bot,\top)$ such that
    $(H,\leq,\wedge,\vee,\bot,\top)$ is a bounded lattice
    and $\Rightarrow$ is an implication operator.

    A Boolean algebra is a structure $(B,\leq,\wedge,\vee,\Rightarrow,\bot,\top)$ such that
    $(B,\wedge,\vee,\Rightarrow,\bot,\top)$ is a Heyting algebra
    and $\Rightarrow$ satisfies the law of excluded middle $a\vee \neg a=\top$.
\end{definition}

\begin{definition}[Category of Heyting algebras]
    The category of Heyting algebras $\mathbf{Heyt}$ is a category such that
    \begin{itemize}
        \item[objects] The objects of $\mathbf{Heyt}$ are the Heyting algebras
        \item[morphisms] The morphisms of $\mathbf{Heyt}$ are the maps that preserve the Heyting algebra structure:
        \begin{gather*}
            f(a\wedge b)=f(a)\wedge f(b)\\
            f(a\vee b)=f(a)\vee f(b)\\
            f(a\Rightarrow b)=f(a)\Rightarrow f(b)\\
            f(\bot)=\bot\\
            f(\top)=\top
        \end{gather*}
    \end{itemize}
\end{definition}
\section{Functors}
Often in mathematics, we are interested in the relationship between objects of different type.
For instance, we might observe that every object $X$ of the category of sets $\mathbf{Set}$
has a corresponding Heyting algebra $(\mathcal{P}X,\subseteq,\cap,\cup,(X\setminus A),\emptyset,X)$
(where $A\Rightarrow B:=(X\setminus A)\cup B$).

We want to formalize the ability to map objects of one category to 
objects of another category.
The mapping should respect the structure of the category.
That is, whenever $A\cong B$ in the base category,
the mapping should map $A$ and $B$ to isomorphic objects in the 
target category.
\begin{definition}[Functor]
    Given two categories $\mathcal{C}$ and $\mathcal{D}$, a functor $F:\mathcal{C}\to\mathcal{D}$ is a pair
    of maps $F_0:\mathcal{C}_0\to\mathcal{D}_0$ and $F_1:\mathcal{C}_1\to\mathcal{D}_1$ such that
    \begin{itemize}
        \item[Respect source and target] Any arrow $f:A\to B$ in $\mathcal{C}$ is mapped to an 
        arrow $F_1(f):F_0(A)\to F_0(B)$ in $\mathcal{D}$
        \item[Respect identity] For any $A\in\mathcal{C}_0$, $F_0(\mathbf{1}_A)=\mathbf{1}_{F_0(A)}$
        \item[Respect composition] For any $f:A\to B$, $F_1(f):F_0(A)\to F_0(B)$
        and for any $f:A\to B$ and $g:B\to C$, $F_1(g\circ f)=F_1(g)\circ F_1(f)$
    \end{itemize}
\end{definition}
One can show that:
\begin{lemma}[Functors preserve isomorphism]
    Given two categories 
    $\mathcal{C}$ and $\mathcal{D}$ and a 
    functor $F:\mathcal{C}\to\mathcal{D}$,
    for any two objects $A,B\in\mathcal{C}_0$,
    if $A\cong B$, then $F_0(A)\cong F_0(B)$.
\end{lemma}
The first example of a functor is the the identity functor $\mathbf{1}_{\mathcal{C}}:\mathcal{C}\to\mathcal{C}$
which maps every object and every arrow to themselves.

As functors are maps between categories, we can ask whether
there is a category of categories with functors as morphisms.
One can easily verify that the composition of two functors $G\circ F$
(understood in the natural way as the pair of maps $G_0\circ F_0$ and $G_1\circ F_1$) is a functor in itself.
The composition is also associative.
Meaning that
\begin{definition}[Category of categories]
    The category of categories $\mathbf{Cat}$ is a category such that
    \begin{itemize}
        \item[objects] The objects of $\mathbf{Cat}$ are the categories
        \item[morphisms] The morphisms of $\mathbf{Cat}$ are the functors
    \end{itemize}
\end{definition}

\begin{remark}
    In the previous section we saw that we can consider the collection of all Heyting algebras as a category
    but we can also consider every Heyting algebra as a category in itself.
    In this case, we forget the operators
    and only consider the elements and the order relation.

    Thus Heyting algebras form a subcategory of the category of categories.
    We shall denote it as $\mathbf{Heyt}_{\leq}$.
    Notice that the morphisms of $\mathbf{Heyt}_{\leq}$,
    that is the functors between the categories, are 
    monotone maps. The morphisms do not necessarily
    preserve the operators of the Heyting algebra.
\end{remark}

\section{Natural Transformations and Adjunctions}

The isomorphisms of $\mathbf{Cat}$
are functors $F$ such that there exists a functor 
$G$ such that $G\circ F=\mathbf{1}_{\mathcal{C}}$
and $F\circ G=\mathbf{1}_{\mathcal{D}}$.
This proves to be a very restrictive notion of isomorphism.
Afterall, the purpose of category theory is to study properties 
of objects up to isomorphism. That is,
it should suffice that 
for any object $A$ in a category $\mathcal{C}$,
we have that $G\circ F(A)\cong A$
and for any object $B$ in a category $\mathcal{D}$,
we have that $F\circ G(B)\cong B$.
When this is the case, we say that $\mathcal{C}$ and $\mathcal{D}$ 
are equivalent categories and denote it by 
$\mathcal{C}\simeq\mathcal{D}$.

Meaning that we have a family of isomorphism arrows
$\{\eta_A:A\to G\circ F(A)\}_{A\in\mathcal{C}_0}$
and
$\{\epsilon_B:F\circ G(B)\to B\}_{B\in\mathcal{D}_0}$.
We would also like that the
arrows agree with the choice of morphisms. That is
for any arrow $f:X\to Y$ in $\mathcal{C}$,
we have that $G\circ F(f)\circ \eta_X=\eta_Y\circ f$. This can be
represented using what is called a commuting diagram:
\[\begin{tikzcd}
	X && {GF(X)} \\
	\\
	Y && {GF(Y)}
	\arrow["{GF(f)}"{description}, from=1-3, to=3-3]
	\arrow["f"{description}, from=1-1, to=3-1]
	\arrow["{\eta_Y}"{description}, from=3-1, to=3-3]
	\arrow["{\eta_X}"{description}, from=1-1, to=1-3]
\end{tikzcd}\]
and likewise we wish a similar diagram to commute for $FG$.

This construction can actually be understood as a morphism between functors
$FG$ and $\mathbf{1}_{\mathcal{C}}$.
\begin{definition}[Natural transformation]
    Given two functors $F,G:\mathcal{C}\to\mathcal{D}$,
    a natural transformation $\eta:F\to G$ is a family of arrows
    $\{\eta_A:F(A)\to G(A)\}_{A\in\mathcal{C}_0}\subseteq\mathcal{D}_1$ 
    such that for any arrow $f:A\to B$ in $\mathcal{C}$,
    the following diagram commutes:
    \[\begin{tikzcd}
        {F(A)} && {G(A)} \\
        \\
        {F(B)} && {G(B)}
        \arrow["{F(f)}"{description}, from=1-1, to=3-1]
        \arrow["{\eta_A}"{description}, from=1-1, to=1-3]
        \arrow["{\eta_B}"{description}, from=3-1, to=3-3]
        \arrow["{G(f)}"{description}, from=1-3, to=3-3]
    \end{tikzcd}\]
\end{definition}
One can easily verify that the family of 
identity arrows $\{\mathbf{1}_{F(A)}\}_{A\in\mathcal{C}_0}$
is an identity natural transformation,
that the component wise composition of natural transformations
is a natural transformation and that this operation is assosiative.
Meaning that:
\begin{definition}[Category of functors]
    For any two categories $\mathcal{C}$ and $\mathcal{D}$,
    there exists a category $[\mathcal{C},\mathcal{D}]$ such that
    \begin{itemize}
        \item[objects] The objects of $[\mathcal{C},\mathcal{D}]$ are the functors $F:\mathcal{C}\to\mathcal{D}$
        \item[morphisms] The morphisms of $[\mathcal{C},\mathcal{D}]$ are the natural transformations
    \end{itemize}
\end{definition}
an equivalence of categories is then
a pair of
functors $F:\mathcal{C}\to\mathcal{D}$ 
and $G:\mathcal{D}\to\mathcal{C}$
such that
$G\circ F\cong \mathbf{1}_{\mathcal{C}}$
and $F\circ G\cong \mathbf{1}_{\mathcal{D}}$.

We already covered one sense in which equivalences preserve
the structure of the category.
\begin{definition}[Essential surjection]
    Given two categories $\mathcal{C}$ and $\mathcal{D}$,
    a functor $F:\mathcal{C}\to\mathcal{D}$ is essentially surjective 
    if for every $B\in\mathcal{D}_0$, there exists an 
    $A\in\mathcal{C}_0$
    such that $F(A)\cong B$.
\end{definition}

However, in categories we also wish to preserve the structure of 
the arrows of the category.
\begin{definition}[Faithfulness and Fullness]
    Given two categories $\mathcal{C}$ and $\mathcal{D}$,
    a functor $F:\mathcal{C}\to\mathcal{D}$ is faithful if for every $A,B\in\mathcal{C}_0$,
    the map $F_1:\mathcal{C}(A,B)\to\mathcal{D}(F(A),F(B))$ is injective.
    A functor $F:\mathcal{C}\to\mathcal{D}$ is full if for every $A,B\in\mathcal{C}_0$,
    the map $F_1:\mathcal{C}(A,B)\to\mathcal{D}(F(A),F(B))$ is surjective.
\end{definition}

This allows us to express what it means for two categories to be equivalent:
\begin{lemma}[Equivalent categories]
    Given two categories $\mathcal{C}$ and $\mathcal{D}$,
    $\mathcal{C}\simeq\mathcal{D}$ if and only if there exists a functor 
    $F:\mathcal{C}\to\mathcal{D}$
    that is essentially surjective, faithful and full.
\end{lemma}

We wish now to weaken further the notion of equivalence.
For that we need to introduce the notion of universal constructions.

\subsection{Universal Constructions}
In \ref{sec:cat_alg_logic} we saw that the conjunction and 
disjunction of two elements
can be understood as a situation in which 'every object that exhibit a
certain arrow pattern has an arrow to/from a unique object'.
This suggests that there is a certain category of
'objects that exhibit a certain arrow pattern' 
that has a terminal/initial object.

\begin{definition}[Diagram]
    Given a category $\mathcal{C}$ and a category
    $\mathcal{D}$, a diagram $D$ of 
    shape $\mathcal{D}$ in $\mathcal{C}$
    is a functor $D:\mathcal{D}\to\mathcal{C}$.
\end{definition}

\begin{definition}[Cone]
    Given a diagram $D$ of shape $\mathcal{D}$ in $\mathcal{C}$,
    a cone over $D$ is a pair $(A,\eta)$ where $A\in\mathcal{C}_0$
    and $\eta$ is a 
    natural transformation $\{\eta_X:A\to D(X)\}_{X\in\mathcal{D}_0}$.
\end{definition}

Given two cones $(A,\eta)$ and $(B,\epsilon)$ over functor $D$,
we can define a morphism between them as a map $f:A\to B$ such that
for any $X\in\mathcal{D}_0$, $\epsilon_X\circ f=\eta_X$.
This defines a category of cones over $D$.
If that category has a terminal object, 
then we say that the diagram $D$ has a limit.
If that category has an initial object, 
then we say that the diagram $D$ has a colimit.

\begin{example}[Product and coproduct]
    Let $\mathcal{D}$ be the category with two objects $0$ and $1$ 
    and no arrows between them.
    A diagram of shape $\mathcal{D}$ in a category $\mathcal{C}$
    is just a pair of objects $A,B\in\mathcal{C}_0$.
    A cone over this diagram is an object $C\in\mathcal{C}_0$
    and two maps $\pi_1:C\to A$ and $\pi_2:C\to B$.
    The limit of this diagram is the product of $A$ and $B$.
    The colimit of this diagram is the coproduct of $A$ and $B$.
\end{example}

Once we introduced natural transformation,
we can define an even weaker notion of similarity between two categories.
