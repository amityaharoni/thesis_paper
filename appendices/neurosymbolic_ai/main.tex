
% Appendix A File

\refstepcounter{chapter}%
\chapter*{\thechapter \quad Neuro-symbolic AI}
\label{appendixB}

Description logics are primarly used in the field of artifical intelligence in the context of knowledge base embeddings.
Knowledge base embeddings are methods of representing a knowledge base in some euclidean space $\mathbb{R}^n$
in such a way that some structural properties of the knowledge base are preserved. The embeddings are then used
for different knowledge representation and Natural Language Processing tasks \cite{SurveryKG}:

% \begin{itemize}
%     \item \textbf{Knowledge Base Completion:} Applying the embeddings to a knowledge base allows for the
%     prediction of missing facts in the knowledge base.
%     As the representation space is a euclidean space, the prediction of missing facts can be done using
%     a scoring function that measures the similarity between two entities in the knowledge base.
%     This allows for the prediction of facts that are implicit in the knowledge base as well as 
%     the prediction of facts that are not present in the knowledge base but are likely to be true.
%     \item \textbf{Entity and Relation Extraction:} Applying the embeddings to a knowledge base allows for
%     the usage of the knowledge base as features in Natural Language Processing (NLP) tasks.
%     This allows for the prediction of entities and relations in text.
% \end{itemize}

A knowledge base embedding is usually constructed through a learning AI algorithm.