
% Appendix A File

\refstepcounter{chapter}%
\chapter*{\thechapter \quad Neuro-symbolic AI}
\label{appendixB}
In this appendix, we provide a brief introduction to neuro-symbolic AI in relation to knowledge bases.
We would start by providing a motivation for the usage of knowledge bases in AI and then we would
provide a brief introduction to the Neural Neutwork algorithms (NNs) that use knowledge bases.

\input{appendices/neurosymbolic_ai/motivation.tex}
\section{The Goal of Supervised Learning}
To describe the training algorithm of NNs, it is important to first describe the task 
that NNs are trying to solve.
Suppose that there is some ground truth $\mathcal{G}:\dlconcepts\times\dlconcepts\to[0,1]$.
This ground truth represents the relationship between the concepts in the knowledge base.
If $\mathcal{K}\vdash C\subseteq D$ then $\mathcal{G}(C,D)=1$ and if $\mathcal{K}\vdash C\not\subseteq D$ then $\mathcal{G}(C,D)=0$.
As mentioned in \ref{item:kb_completion}, one goal of NNs is
to diagnose possible relations between concepts that could not be directly inferred (or potentially require a lot of computation to infer) from the knowledge base.
Hence the ground truth $\mathcal{G}$ should also be able to assign a confidence value between $0$ and $1$ to concepts that are not related.
If $\mathcal{K}$ cannot prove any relationship between $C$ and $D$ then $\mathcal{G}(C,D)$ assigns a value between $0$ and $1$.
Note that we do not assume to have access to the ground truth $\mathcal{G}$.
We only assume that there is some ground truth $\mathcal{G}$ that we would like to approximate.
The goal of the NN is to generate an embedding function $E$ such that $E$ is as close to $\mathcal{G}$ as possible.

This allows us to assess the quality of the embedding function by measuring the expected square error of the embedding function from the ground truth.
\[
    \mathcal{L}:=\mathbb{E}_{C,D\in\dlconcepts}[(E(C,D)-\mathcal{G}(C,D))^2]
\]
This function is what we call the loss function.

Thus, given a class of possible embedding classes $\mathcal{E}$ (called the hypothesis class), the goal of the NN is to find an embedding function $E\in\mathcal{E}$
such that
\[
    E=\argmin_{E'\in\mathcal{E}}\mathcal{L}_{\mathcal{G}}(E')    
\]


However, due to computational constraints, 
training is rarely done on all possible concept pairs. To train our model, we sample a set of input-output
data points $\mathbb{D}=\{((C_i,D_i),\delta_{\mathcal{K}\vdash C_i\sqsubseteq D_i})\}$
where $\delta$ is the Kronecker delta function.
Thus what we actually compute is
\[
    \mathcal{L}_{\mathbb{D}}:=\mathbb{E}_{((C,D),l)\in\mathbb{D}}[(E(C,D)-l)^2]
\]
This function is what we call the empirical loss function. The goal of the NN is to minimize the empirical loss function
with the hope that the small dataset models sufficiently well the loss over the ground truth.
This is called the \it{Empirical Risk Minimization} (ERM) principle.

In \cite{AgnosticPAC} it has been shown that if the hypothesis class $\mathcal{E}$ is finite, 
then there exists a polynomial time learning algorithm
that can get sufficiently close to the ERM solution by minimizing the empirical loss function
and increasing the amount of data points in $\mathbb{D}$.

\section{Artificial Neural Networks}
We saw that for any
finite hypothesis class $\mathcal{H}$ it is possible to find a polynomial time 
algorithm that can get sufficiently close to the ERM solution by 
minimizing the empirical loss function.

Given a sample space $X$,
we can think of the hypothesis class $\mathcal{H}$ as
a function $f:\Theta\to\mathcal{H}$
where $\Theta$ is a set of hyperparameters 
that fix a single hypothesis in $\mathcal{H}$.
If $f$ is continuous, then we can find the ERM solution by
taking the derivative of the empirical loss function with respect to $\Theta$ and setting it to $0$.
\[
    \frac{\partial}{\partial\Theta}\mathcal{L}_{\mathbb{D}}(f(\Theta))=0    
\]
We can then use algebraic methods to solve for $\Theta$. 
This approach relies on having a nicely behaved function $f$ for which we can solve
the differential equation above.
However in practice it is difficult to identify the hypothesis class $\mathcal{H}$ 
with a nicely behaving function $f$.
We can however identify a function $f$ that can approximate many hypothesis classes
and use a version of gradient descent to find the ERM solution.
The function $f$ that we would use is called a neural network and the learning
algorithm is called stochastic gradient descent.

However, we wish to find a methodological way to generate a learning algorithm
that can get sufficiently close to the ERM solution for any hypothesis class.
This is the goal of the field of \it{machine learning}.
Neural networks are a family of machine learning algorithms that have been shown to be
effective in practice for a wide range of tasks. 
\subsection{Architecture}
\begin{definition}[Neural Network Architecture]
    A neural network architecture is a tuple $(G,\Sigma)$ where
    \begin{enumerate}
        \item $G=(V,E)$ is an acyclic directed graph.
        \item $V$ is composed of layers $V=\bigsqcup^L_{l=0} V_l$.
        \item For every edge $e\in E$, there exists an $l\in[L-1]$ such that 
        the source node of $e$ is in layer 
        $l$ and the target node is in layer $l+1$.
        \item $\Sigma$ is a collection of non-polynomial
        functions
        $\{\sigma_{i}:\mathbb{R}^{\lvert V_{l}\rvert}\to\mathbb{R}^{\lvert V_{l}\rvert}\}_{l\in [L-1]}$.
    \end{enumerate}
    $L$ is called the depth of the neural network,
    for any $l\in[L]$, we call $V_l$ the $l$-th layer of the neural network.
    We call $V_0$ the input layer and $V_L$ the output layer.
    The layers in between are called hidden layers.
    The size of the $l$-th layer $\lvert V_l \rvert$ is called the width of the layer.

    For any $l\in[L]$, we call $\sigma_l$ the activation function of the $l$-th layer.
\end{definition}
\begin{definition}[Neural Network]
    Given a neural network architecture $(G,\Sigma)$,
    a neural network consists of
    \begin{enumerate}
        \item A set of matrices $W=\{W_l\}_{l=1}^L$ where $W_l$ is a matrix of size
        $\lvert V_l\rvert\times\lvert V_{l+1}\rvert$. Furthermore, $W_{l,i,j}=0$ if
        $(v_{l,i},v_{l+1,j})\not\in E$.
        \item A set of vectors $B=\{B_l\}_{l=1}^L$ where $B_l$ is a vector of size
        $\lvert V_{l+1}\rvert$.
    \end{enumerate}
    We call $W$ the weight matrices and $B$ the bias vectors.
\end{definition}
Given a neural network $(G,\Sigma)$ with a set of weight matrices $W$ and bias vectors $B$,
we define the set of preactivation functions $Z=\{z_l:\mathbb{R}^{\lvert V_l\rvert}\to\mathbb{R}^{\lvert V_{l+1}\rvert}\}_{l\in[L-1]}$ where
\begin{align*}
    z_{l}(\mathbf{x})&= \sigma_{l}(\mathbf{x})\cdot W_{l+1} +B_{l+1}
\end{align*}

And the function represented by the neural network is the concatenation of the preactivation functions:
\[
    f_{W,B}(\mathbf{x})=\bigcirc_{l=0}^{L-1} z_l(\mathbf{x})
\]

The pair of weights and biases $\Theta=(W,B)$ is called the parameters of the neural network.

The main reason for the popularity of neural networks is the universal approximation theorem.
\begin{theorem}[Universal Approximation Theorem]
    Let $\sigma:\mathbb{R}\to\mathbb{R}$ be a non-constant, bounded, and continuous function.
    Let $I\subseteq\mathbb{R}^n$ be a compact set.
    Then for any $\epsilon>0$ and any continuous function $f:I\to\mathbb{R}$,
    there exists an architecture $(G,\Sigma)$ 
    and a set of weights $W$ and biases $B$ such that
    \[
        \lvert f(x)-f_{W,B}(x)\rvert<\epsilon
    \]
    for all $x\in I$.
\end{theorem}

\subsection{Training}
It is now left to find a method to find the parameters $\Theta$ of a neural network
that minimize the empirical loss function.