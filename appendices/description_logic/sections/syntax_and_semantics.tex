\section{$\mathcal{ALC}$ Syntax and Semantics}
\label{appendixB:section:dl_syntax_semantics}

Following the introduction, we formalize the above ideas in a more precise manner.
We begin by introducing the syntax and semantics of one of the most expressive DLs, namely the Attributive Concept Language with Complements DL ($\mathcal{ALC}$).

\subsection{Syntax}
\label{dl_syntax_semantics:subsection:dl_syntax}

In logic, the syntax of a language is the set of well-formed formulas of the language.
The syntax of a DL is defined by a signature and a grammar.
The signature defines the set of primitive names of the DL (e.g., concept names, role names, and individual names). 
The grammar defines how to combine the primitive names to form complex names (e.g., concepts and roles).
The set of complex names is called the set of concepts and roles of the DL.

This section would rely on the construction found in \cite{DLhandbook}.

\begin{definition}[Signature]
    Given a description logic (DL), a
     \emph{signature} $\Sigma$ is a triple $\Sigma = (N_\dlconcepts, \dlroles, \dlindividuals)$,
    where $N_\dlconcepts$, $\dlroles$, and $\dlindividuals$ are pairwise disjoint sets of \emph{concept names}, \emph{roles}, and \emph{individuals}, respectively.
\end{definition}

A description logic is defined by the set of constructors it supports.
The constructors of $\mathcal{ALC}$ are $\exists$, $\forall$, $\sqcap$, $\sqcup$, $\neg$, $\top$, and $\bot$.

\begin{definition}[$\mathcal{ACL}$ Concept]
    Given a signature $\Sigma$, 
    a \emph{$\mathcal{ALC}$-concept} $C$ is defined recursively by the following grammar:
    \begin{align*}
        C &::= A \mid \top \mid \bot \mid D \sqcap E \mid D \sqcup E \mid \neg D \mid \exists R.D \mid \forall R.D
    \end{align*}
    where $A \in N_\dlconcepts$, $R \in \dlroles$ and $D, E$ are previously defined $\mathcal{ALC}$-concepts.
    The set of $\mathcal{ALC}$-concepts over a signature $\Sigma$ is denoted by $\dlconcepts$ and is defined as the smallest 
    set that contains $N_\dlconcepts \subseteq \dlconcepts$ and every concept $C$ that can be recursively constructed 
    from the primitive concepts in $N_\dlconcepts$ using the above grammar.
\end{definition}

Some DL languages also allow the use of \emph{nominals} in concepts.
A nominal is a concept that is interpreted as a singleton set of individuals.
We say that a DL supports nominals if it supports the constructor $\{a\}$, where $a \in \dlindividuals$.
That is, if every concept is recursively constructed from a set of constructors which also includes $\{a\}$.

\subsection{Semantics}
\label{dl_syntax_semantics:subsection:dl_semantics}

In logic, the semantics of a language is the set of interpretations of the language.
An interpretation is a mapping from the primitive names of the language to elements of a domain.
The semantics of a DL is defined by a fixed domain and an interpretation.
The fixed domain is the set of elements that can be used to interpret the primitive names of the DL.
The interpretation is a mapping from the primitive names of the DL to elements of the fixed domain.
The interpretation of a complex name is defined in terms of the interpretation of its subnames.
The standard semantics of $\mathcal{ALC}$ treats concepts as subsets of the fixed domain and roles as binary relations over the fixed domain.

\begin{definition}[Interpretation]
    Given a signature $\Sigma$, an \emph{interpretation} $\mathcal{I}$ is a triple $\mathcal{I} = (\Delta^\mathcal{I}, \cdot^\mathcal{I}, \cdot^\mathcal{I})$,
    where $\Delta^\mathcal{I}$ is a non-empty set called the \emph{domain} of $\mathcal{I}$,
    $\cdot^\mathcal{I}:\dlconcepts\to\mathcal{P}\Delta^\mathcal{I}$ is 
    an \emph{interpretation function} that maps each concept $C \in \dlconcepts$ to a subset 
    $C^\mathcal{I} \subseteq \Delta^\mathcal{I}$,
    and $\cdot^\mathcal{I}:\dlroles\to\mathcal{P}(\Delta^\mathcal{I}\times\Delta^\mathcal{I})$ is
    an \emph{interpretation function} that maps each role name $R \in N_\dlroles$ to a binary relation $R^\mathcal{I} \subseteq \Delta^\mathcal{I} \times \Delta^\mathcal{I}$.

    The interpretation function $\cdot^\mathcal{I}$ must satisfies the following conditions:
    \begin{itemize}
        \item $\top^\mathcal{I} = \Delta^\mathcal{I}$ and $\bot^\mathcal{I} = \emptyset$.
        \item For every $a \in \dlindividuals$, $a^\mathcal{I} \in \Delta^\mathcal{I}$.
        \item $(C \sqcap D)^\mathcal{I} = C^\mathcal{I} \cap D^\mathcal{I}$ and $(C \sqcup D)^\mathcal{I} = C^\mathcal{I} \cup D^\mathcal{I}$.
        \item $(\neg C)^\mathcal{I} = \Delta^\mathcal{I} \setminus C^\mathcal{I}$.
        \item $(\exists R.C)^\mathcal{I} = \{x \in \Delta^\mathcal{I} \mid \exists y \in \Delta^\mathcal{I}.(x, y) \in R^\mathcal{I} \land y \in C^\mathcal{I}\}$.
        \item $(\forall R.C)^\mathcal{I} = \{x \in \Delta^\mathcal{I} \mid \forall y \in \Delta^\mathcal{I}.(x, y) \in R^\mathcal{I} \implies y \in C^\mathcal{I}\}$.
    \end{itemize}
\end{definition}

If a DL supports nominals, then the interpretation function $\cdot^\mathcal{I}$ must also satisfy the following condition:
\begin{itemize}
    \item $(\{a\})^\mathcal{I} = \{a^\mathcal{I}\}$.
\end{itemize}

\subsection{Knowledge Base}
\label{dl_syntax_semantics:subsection:dl_kb}

Description logics are used to represent knowledge about a domain of interest (e.g., a medical domain).
The knowledge about a domain of interest is represented by a knowledge base.
A knowledge base is a set of axioms.
An axiom is a statement that describes a relation between concepts, roles, and individuals.
The semantics of a knowledge base is defined in terms of the semantics of its axioms.

\begin{definition}[TBox and Subsumption]
    Given a signature $\Sigma$, a \emph{TBox} $\mathcal{T}$ is a finite set of \emph{subsumption axioms} of the form $C \sqsubseteq D$,
    where $C, D \in \dlconcepts$.
\end{definition}

\begin{notation}[Definition]
    Given a signature $\Sigma$, a \emph{definition}, denoted as $C \equiv D$, is an abbreviation
    for the subsumption axioms $C \sqsubseteq D$ and $D \sqsubseteq C$.
\end{notation} 

\begin{definition}[ABox and Assertion]
    Given a signature $\Sigma$, an \emph{ABox} $\mathcal{A}$ is a finite set of \emph{assertion axioms} of the form $C(a)$ or $R(a, b)$,
    where $C \in \dlconcepts$, $R \in \dlroles$, and $a, b \in \dlindividuals$.
\end{definition}

\begin{definition}[Satisfiability]
    Given an interpretation $\mathcal{I}$, we say that
    \begin{itemize}
        \item $\mathcal{I}$ \emph{satisfies} a concept $C \in \dlconcepts$ if $C^\mathcal{I} \neq \emptyset$.
        \item $\mathcal{I}$ \emph{satisfies} a role $R \in \dlroles$ if $R^\mathcal{I} \neq \emptyset$.
        \item $\mathcal{I}$ \emph{satisfies} a subsumption axiom $C \sqsubseteq D$ if $C^\mathcal{I} \subseteq D^\mathcal{I}$.
        \item $\mathcal{I}$ \emph{satisfies} an assertion axiom $C(a)$ if $a^\mathcal{I} \in C^\mathcal{I}$.
        \item $\mathcal{I}$ \emph{satisfies} an assertion axiom $R(a, b)$ if $(a^\mathcal{I}, b^\mathcal{I}) \in R^\mathcal{I}$.
        \item $\mathcal{I}$ \emph{satisfies} a TBox $\mathcal{T}$ if $\mathcal{I}$ satisfies every subsumption axiom in $\mathcal{T}$.
        \item $\mathcal{I}$ \emph{satisfies} an ABox $\mathcal{A}$ if $\mathcal{I}$ satisfies every assertion axiom in $\mathcal{A}$.
    \end{itemize}
\end{definition}

\begin{definition}[Knowledge Base]
    Given a signature $\Sigma$, a \emph{knowledge base} $\mathcal{K}$ is a pair $\mathcal{K} = (\mathcal{T}, \mathcal{A})$,
    where $\mathcal{T}$ is a TBox and $\mathcal{A}$ is an ABox.
\end{definition}

\begin{definition}[Model]
    Given a knowledge base $\mathcal{K} = (\mathcal{T}, \mathcal{A})$ and an interpretation $\mathcal{I}$,
    we say that $\mathcal{I}$ is a \emph{model} of $\mathcal{K}$ if $\mathcal{I}$ satisfies $\mathcal{T}$ and $\mathcal{A}$.

    We say that a knowledge base $\mathcal{K}$ is
consistent if it has a model.
\end{definition}

\begin{notation}[Satisfiability]
    We will write $\mathcal{I}\vDash\mathcal{L}$ if $\mathcal{I}$ satisfies $\mathcal{L}$.
\end{notation}

Under this semantics, a query $q$ is entailed by a knowledge base $\mathcal{K}$ if $q$ is true in all models of $\mathcal{K}$.