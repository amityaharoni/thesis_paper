\section{$\mathcal{EL}++$ Syntax and Semantics}
\label{sec:existential_language}
For most applications, $\mathcal{ALC}$ 
is too computationally expensive to be used in practice. 
For example, the subsumption problem in 
$\mathcal{ALC}$ is 
$\pspace$-complete 
\cite{ALCcomplexity}.
To address this issue,
a number of tractable DLs have been proposed.

One of the most popular tractable DLs is the
$\mathcal{EL}$ family of DLs.
The $\mathcal{EL}$ family of DLs is a subset of the $\mathcal{ALC}$ family of DLs
that contains only the constructors $\exists$, $\sqcap$, $\neg$ and $\top$.

\begin{definition}[EL Concept]
    Given a signature $\Sigma$, an \emph{$\mathcal{EL}$-concept} $C$ is defined recursively by the following grammar:
    \begin{align*}
        C &::= A \mid \bot \mid D \sqcap E \mid \neg D \mid \exists R.D
    \end{align*}
    where $A \in N_\dlconcepts$, $R \in \dlroles$ and $D, E$ are previously defined $\mathcal{EL}$-concepts.
    The set of $\mathcal{EL}$-concepts over a signature $\Sigma$ is denoted by $\dlconcepts$ and is defined as the smallest 
    set that contains $N_\dlconcepts \subseteq \dlconcepts$ and every concept $C$ that can be recursively constructed 
    from the primitive concepts in $N_\dlconcepts$ using the above grammar.
\end{definition}
Interpretation of $\mathcal{EL}$-concepts is similar to the interpretation of $\mathcal{ALC}$-concepts.
The main advantage of $\mathcal{EL}$ is that subsumption in $\mathcal{EL}$ is polynomial time \cite{ELcomplexity}.
In \cite{EL++complexity} it was shown that adding the constructor $\bot$ and nominality does not increase the complexity of subsumption.
The resulting DL is called $\mathcal{EL}^{++}$.

\begin{definition}[$\mathcal{EL}^{++}$ Concept]
    Given a signature $\Sigma$, an \emph{$\mathcal{EL}^{++}$-concept} $C$ is defined recursively by the following grammar:
    \begin{align*}
        C &::= A \mid \top \mid \bot \mid D \sqcap E \mid \neg D \mid \exists R.D \mid \{a\}
    \end{align*}
    where $A \in N_\dlconcepts$, $R \in \dlroles$ and $D, E$ are previously defined $\mathcal{EL}^{++}$-concepts.
    The set of $\mathcal{EL}^{++}$-concepts over a signature $\Sigma$ is denoted by $\dlconcepts$ and is defined as the smallest 
    set that contains $N_\dlconcepts \subseteq \dlconcepts$ and every concept $C$ that can be recursively constructed 
    from the primitive concepts in $N_\dlconcepts$ using the above grammar.
\end{definition}